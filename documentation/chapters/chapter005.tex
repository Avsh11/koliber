\chapter{Instrukcja uruchomieniowa}
Aby poprawnie uruchomić skrypty a co za tym idzie kompletną aplikację, należy przeprowadzić następujące kroki:
\begin{enumerate}
    \item Zainstalować środowisko \textbf{VirtualBox}, najlepiej w najnowszej wersji.
    \item Utworzyć maszynę wirtualną - Ubuntu Server. W tym przypadku wersja \texttt{Ubuntu 24.04.3 LTS}.
    \item Utworzyć klienckie maszyny - w przypadku projektu Windows 11 LTSC oraz Linux Mint.
    \item Plik \texttt{client.py} przesłać do maszyn klienckich np. na pulpit.
    \item Plik \texttt{server.py} przesłać na serwer Ubuntu.
    \item Plik \texttt{server.py} uruchomić na serwerze poprzez komendę \texttt{python3 server.py} - po wykonaniu komendy powinien przywitać nas odpowiedni komunikat.
    \item Plik \texttt{client.py} uruchomić poprzez komendę \texttt{python client.py (Windows) / python3 client.py (Linux)} uprzednio przechodząc w odpowiednią ścieżkę w terminalu - CMD / Bash.
    \item Po wykonaniu wcześniejszych kroków całość powinna działać poprawnie.
\end{enumerate}