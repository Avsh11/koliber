\chapter{Opis założeń projektu}

Celem projektu było stworzenie prostego i anonimowego komunikatora sieciowego przy użyciu języka programowania \textbf{Python}. Program ma za zadanie umożliwić wielu użytkownikom jednoczesną wymianę wiadomości tekstowych w czasie rzeczywistym, wykorzystując arhitekturę wielowątkową również zaimplementowaną używając wspomnianego wcześniej języka \textbf{Python} dobierając wcześniej odpowiednie moduły takie jak \texttt{threading} czy \texttt{socket}. Program ma działać w środowisku rozproszonym, łącząc maszyny o różnych systemach operacyjnych pracując wewnątrz wirtualnej infrastruktury sieciowej.

Podstawowym problemem, który zostanie rozwiązany przez realizację tego projektu, jest zagrożenie prywatności w sieci oraz konieczność udostępniania danych osobowych w tradycyjnych systemach komunikacji. Obecnie wiele krajów zachodnich takich jak chociażby \textit{Wielka Brytania} coraz częściej z wrogością zwraca się do prywatnej wymiany informacji w obawie o to iż użytkownicy będą komunikować między sobą informacje nie będące po drodze narracji rządowej. Projekt inspirując się forum obrazkowym \textit{4chan} rozwiązuje ten problem poprzez brak konieczności zakładania kont oraz wykorzystanie tymczasowych identyfikatorów sesji, które są generowane losowo przez serwer i usuwane natychmiast po rozłączeniu użytkownika z pamięci operacyjnej węzła.

Aby problem został skutecznie rozwiązany, potencjalny zespół musi posiadać wiedzę z zakresu architektury sieciowej \textbf{TCP/IP}, modelu \textbf{ISO/OSI} (w szczególności warstwy transportowej i aplikacji) oraz podstawowej obsługi wątków.

Rozwiązanie problemu przebiegło w kilku zdefiniowanych krokach. W pierwszej kolejności nastąpiło zaprojektowanie topologii sieciowej w środowisku \texttt{VirtualBox}. Kolejnym krokiem było zaimplementowanie logicznej warstwy serwera, odpowiedzialnego za akceptowanie połączeń i przesyłanie komunikatów. Następnie wdrożono obsługę wielu wątków oraz system anonimowego przydzielania identyfikatorów. Końcowym etapem było stworzenie podstawowego interfejsu \textbf{CLI} dla klientów, gdzie Ci będą mogli odróżniać wiadomości swoje od cudzych za pośrednictwem kolorów. Na samym końcu dodano również możliwość przywrócenia starej konwersacji poprzez wpisanie swojego starego ID użytego wcześniej.

\section{Wymagania funkcjonalne}
\begin{itemize}
    \item \textbf{Nawiązywanie połączenia:} Użytkownik musi mieć możliwość połączenia się z serwerem za pomocą protokołu TCP/IP po podaniu adresu IP serwera oraz zdefiniowanego portu (55555) - poziom skryptu \texttt{client.py}.
    \item \textbf{Automatyczne przydzielenie tożsamości:} System musi wygenerować i przypisać każdemu użytkownikowi unikalny, 6-znakowy identyfikator sesji - ID natychmiast po nawiązaniu połączenia przez użytkownika.
    \item \textbf{Wymaiana wiadomości w czasie rzeczywistym:} Wiadomość wpisana przez jednego użytkownika musi zostać natychmiast rozesłana przez serwer do wszystkich pozostałych aktywnych uczestników sesji.
    \item \textbf{Identyfikacja nadawcy:} Każda wiadomość przesłana w sieci musi być opatrzona identyfikatorem ID nadawcy, aby odbiorcy wiedzielo, od kogo pochodzi wiadomość.
    \item \textbf{Powiadomienia systemowe:} Program musi informować wszystkich uczestników o zmianach w strukturze sieci (dołączenia nowego użytkownika albo opuszczenie przez niego sesji).
    \item \textbf{Wizualne odróżnienie komunikatów:} Interfejs musi za pomocą kolorów odróżniać własne wiadomości od wiadomości innych użytokwników oraz komunikatów systemowych.
    \item \textbf{Automatyczne sprzątanie sesji:} Serwer musi automatycznie wykrywać rozłączanie klienta, zamykać powiązany z nim socket i zwalniać zajęte ID do ponownej puli losowania.
    \item \textbf{Przywrócenie konwersacji:} Gdy użytkownik posiada swoje stare ID, może przywrócić treść konwersacji w której uczestniczył.
\end{itemize}
\section{Wymagania niefunkcjonalne}
\begin{itemize}
    \item \textbf{Niezawodność transmisji:} Komunikacja musi opierać się na protokole TCP, aby zagwarantować, że wiadomości dotrą do odbiorców w całości i w poprawnej kolejności.
    \item \textbf{Wielowątkowość:} Serwer musi być zdolny do obsługi wielu gniazd sieciowych (socketów) jednocześnie bez blokowania głownego procesu aplikacji.
    \item \textbf{Wieloplatformowość:} Kod musi być kompatybliny z różnymi systemami operacyjnymi począwszy od \texttt{Linux} aż po sysetmy z rodziny \texttt{Microsoft Windows}.
    \item \textbf{Minimalizm:} Oprogramowanie ma działać wyłącznie w trybie tekstowym CLI minimalizując tym samym zużycie pamięci RAM i CPU.
    \item \textbf{Brak śladu cyfrowego:} Wszystkie dane o użytkownikach i treści wiadomości muszą być przechowywane wyłącznie w pamięci RAM. Brak logowania danych na dysku twardym serwera zapewnia anonimowość i prywatność.
    \item \textbf{Izolacja środowiska:} System musi poprawnie pracować wewnątrz wirtualnej sieci (VirtualBox).
    \item \textbf{Samowystarczalność:} Projekt musi być napisany w czystym języku \textbf{Python} przy użyciu wyłącznie bibliotek standardowych, co elimunuje konieczność instalowania zewnętrznych bibliotek i zależności.
\end{itemize}
