\chapter{Prezentacja działania i testy}

\section{Serwer Ubuntu: komunikaty rozpoczęcia pracy i dołączenia}

Działanie serwera wraz z komunikatami o dołączeniu  się użytkowników są przedstawione na obrazku poniżej - \textbf{rys. \ref{rys 4.1}}
\begin{figure}[H]
    \centering
    \includegraphics[width=0.8\textwidth]{figures/4.1.png}
    \caption{Działanie serwera i komunikat dołączenia użytkownika}\label{rys 4.1}
\end{figure}

\section{Serwer Ubuntu: komunikaty rozłączania się użytkowników}
Działanie serwera wraz z komunikatami o dołączeniu  się użytkowników i ich następnym rozłączeniu są przedstawione na obrazku poniżej - \textbf{rys. \ref{rys 4.2}}
\begin{figure}[H]
    \centering
    \includegraphics[width=0.8\textwidth]{figures/4.2.png}
    \caption{Rozłączenie użytkownika - perspektywa serwera}\label{rys 4.2}
\end{figure}

\section{Działanie aplikacji z perspektywy maszyn klienckich}
\subsection{Przesyłanie wiadomości i komunikat dołączenia}
Działanie aplikacji po stronie użytkowników na maszynach \texttt{Windows 11 LTSC} i \texttt{Linux Mint}, gdzie widoczne jest przesyłanie wiadomości oraz komunikat o dołączeniu użytkownika widoczne jest na rysunku poniżej - \textbf{rys. \ref{rys 4.3}}
\begin{figure}[H]
    \centering
    \includegraphics[width=0.8\textwidth]{figures/4.3.png}
    \caption{Przekazywanie wiadomości i dołączenie użytkownika}\label{rys 4.3}
\end{figure}

\subsection{Rozłączenie się - komunikat}
Komunikat informujący o rozłączeniu się jednego z użytkowników widoczny jest na obrazku poniżej - \textbf{rys. \ref{rys 4.4}}
\begin{figure}[H]
    \centering
    \includegraphics[width=0.8\textwidth]{figures/4.4.png}
    \caption{Rozłączenie użytkownika - komunikat}\label{rys 4.4}
\end{figure}

\subsection{Przywracanie historii - treść}
Oboje użytkowników zaczęło nową sesję. Natomiast tylko użytkownik o ID \texttt{ACA6MQ} zapamiętał swoje ID i użyje je do przywrócenia konwersacji. Warto zobaczyć co dzieje się na terminalu maszyny \texttt{Linux Mint}, która rozpoczęła nową sesje. Wynik operacji widoczny jest na obrazku poniżej - \textbf{rys. \ref{rys 4.5}}
\begin{figure}[H]
    \centering
    \includegraphics[width=0.8\textwidth]{figures/4.5.png}
    \caption{Przywrócenie historii, gdzie konkretny ID uczestniczył}\label{rys 4.5}
\end{figure}

\subsection{Dostępność historii dla nowych użytkowników - zachowanie}
Zachowanie to przedstawione jest na obrazku powyżej - \textbf{rys. \ref{rys 4.5}}
\subsection{Zabezpieczenie - co jeśli ktoś jest nowy i chce przywrócić sesję?}
W przypadku, gdy ktoś uzna, że nie chce zacząć nowej sesji i wpisze nieistniejące ID, zostanie mu przydzielone nowe ID losowe co spowoduje ten sam proces co przy stworzeniu nowej sesji (Opcja TAK). Efekt widoczny jest na obrazkach poniżej - \textbf{rys. \ref{rys 4.6}, rys. \ref{rys 4.7}, rys. \ref{rys 4.8}}
\begin{figure}[H]
    \centering
    \includegraphics[width=0.8\textwidth]{figures/4.6.png}
    \caption{Czy chcę rozpocząć nową sesję? (N)}\label{rys 4.6}
\end{figure}
\begin{figure}[H]
    \centering
    \includegraphics[width=0.8\textwidth]{figures/4.7.png}
    \caption{Podanie nieistniejącego ID}\label{rys 4.7}
\end{figure}
\begin{figure}[H]
    \centering
    \includegraphics[width=0.8\textwidth]{figures/4.8.png}
    \caption{Efekt końcowy - rozpoczęcie nowej sesji z losowymi ID}\label{rys 4.8}
\end{figure}